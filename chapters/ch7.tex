\section{Conclusione}


In questa tesi è stato combinato il mio interesse per il lago di Endine e la passione per la progettazione . Mi è stata data l’occasione di  studiare, in modo approfondito, gli elementi che riguardano la realizzazione del bar; tema che non abbiamo potuto approfondire in classe. Al fine di creare un risultato interessante si è pensato di utilizzare  una struttura dal design non convenzionale, sul quale è stato adattato il progetto del bar. Il progetto è stato progettato interamente in CAD tutta la struttura è stata progettata tridimensionalmente al fine di poterla stampare in 3D. Infatti avere un oggetto tangibile alla fine della progettazione ci a permesso di valutare meglio la qualità della struttura ed individuare possibili difetti. Avere un prototipo in scala del progetto finale ci permette di mostrare al meglio il prodotto finale ad un possibile cliente.

Durante la formazione di apprendistato, della durata di 8 mesi, mi è stato permesso dall'azienda ospitante di utilizzare parte delle mie ore  nella progettazione; permettendomi  di imparare ad utilizzare un programma di disegno: Bsolid. Tutto questo mi ha maggiormente motivato a non terminare il mio percorso formativo e voler continuare con il quinto anno per ampliare maggiormente le mie competenze riguardanti il disegno. Tutto questo mi è necessario per poi poter accedere ad una formazione universitaria orientata al design. \\

\noindent
\cite{maggioli} \cite{endine} \cite{-_di_-_tecnica_here_2017} \cite{commercio} \cite{decreto}  \cite{design}
\cite{banconemis} \cite{bancone}  \cite{celle}
 \cite{caffe} \cite{decreto2} \cite{acqua} \cite{argine} \cite{documenti} \cite{barriere}

