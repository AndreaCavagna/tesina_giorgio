\section{Conclusione}


Con questa mia tesi ho voluto combinare il mio interesse per il lago di Endine e la mia passione per la progettazione .

Mi ha dato l’occasione di  studiare, in modo approfondito, gli elementi che riguardano la realizzazione del bar, tema che non abbiamo potuto approfondire in classe.
Ho voluto inoltre mettermi alla prova utilizzando una struttura di design, dalle forme non convenzionali alle quali ho dovuto adattare il progetto del bar. 
Mi sono inoltre divertito a progettare tridimensionalmente tutta la struttura  per poi realizzarla con la stampante 3D .

Durante la mia formazione di apprendistato, articolo 43 , della durata di 8 mesi, l’azienda  mi ha permesso di impiegare parte delle mie ore in  laboratorio e una parte nella progettazione, permettendomi  anche di imparare ad utilizzare un nuovo programma di disegno chiamato bsolid. 
Tutto questo mi ha maggiormente motivato a voler continuare con il quinto anno il mio percorso di studi, per ampliare maggiormente le mie competenze riguardanti il disegno  e  poter poi accedere al Politecnico di design.    \cite{maggioli} ~ \cite{endine} ~ \cite{-_di_-_tecnica_here_2017} \cite{commercio} \cite{decreto}  \cite{design}
\cite{banconemis} \cite{bancone}  \cite{celle}
 \cite{caffe} \cite{decreto2} \cite{acqua} \cite{argine} \cite{documenti} \cite{barriere}

