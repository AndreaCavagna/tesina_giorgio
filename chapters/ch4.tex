\section{Regolamento Comune}

L’installazione su area pubblica di un chiosco per la vendita di giornali o per la somministrazione al pubblico di alimenti e bevande richiede, oltre all’autorizzazione per l’occupazione di suolo pubblico e a quella per l’attività commerciale anche il permesso di costruire di cui al DPR 380/01.


\subsection{Art. 3 (L) - Definizioni degli interventi edilizi}

1. Ai fini del presente testo unico si intendono per:
d) "interventi di ristrutturazione edilizia", gli interventi rivolti a trasformare gli organismi edilizi mediante un insieme sistematico di opere che possono portare ad un organismo edilizio in tutto o in parte diverso dal precedente. Tali interventi comprendono il ripristino o la sostituzione di alcuni elementi costitutivi dell'edificio, l’eliminazione, la modifica e l'inserimento di nuovi elementi ed impianti. Nell'ambito degli interventi di ristrutturazione edilizia sono ricompresi anche quelli consistenti nella demolizione e ricostruzione con la stessa volumetria di quello preesistente, fatte salve le sole innovazioni necessarie per l'adeguamento alla normativa antisismica nonché quelli volti al ripristino di edifici, o parti di essi, eventualmente crollati o demoliti, attraverso la loro ricostruzione, purché sia possibile accertarne la preesistente consistenza. 

\subsection{Art. 10 (L) - Interventi subordinati a permesso di costruire}

1. Costituiscono interventi di trasformazione urbanistica ed edilizia del territorio e sono subordinati a permesso di costruire:
a) gli interventi di nuova costruzione;
b) gli interventi di ristrutturazione urbanistica;
c) gli interventi di ristrutturazione edilizia che portino ad un organismo edilizio in tutto o in parte diverso dal precedente e che comportino modifiche della volumetria complessiva degli edifici o dei prospetti, ovvero che, limitatamente agli immobili compresi nelle zone omogenee A, comportino mutamenti della destinazione d’uso, nonché gli interventi che comportino modificazioni della sagoma di immobili sottoposti a vincoli ai sensi del decreto legislativo 22 Gennaio 2004, n. 42 e successive modificazioni.